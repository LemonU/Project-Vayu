% !TEX root = ./Requirment.tex
\documentclass{article}
\usepackage[utf8]{inputenc}
\usepackage{url} 
\title{Sections and Chapters}
\author{Christina Korsman,Oussama Saoudi}
\date{ }
  
\begin{document}
  
\maketitle
  
\tableofcontents

\section{Introduction}
\subsection{Purpose}
This document is intended to act as a software requirements 
specification for the program, which will outline the 
requirements that the software must meet and the priority 
of each of said requirements. The intended audience for the document 
is the client requesting the software, who will use it to understand 
the formalized specification for the software project being developed. 
The software developers will also use the document in the development 
of components to ensure that they meet the requirements specified.

\subsection{Scope}
The software being developed is called Project: Vayu.\\
 
 Project: Vayu aims to provide information on natural disasters, 
 particularly information on property damage in sorted order, 
 and geographical locations affected by particular natural disasters. 
 The software will not add new data.\\
  
The objective of the project is to present natural disaster information in forms relevant to
disaster relief organizations and governments. 
This will be done by presenting regions prone to
certain disasters and which regions are most affected by natural 
disasters and their property costs and casualties. The representation can 
be modified depending on the time of year, allowing for resource management of 
relief to be allocated to the right places at the right times of the year.

\subsection{References}
[1]“Disaster Prediction App – Apps on Google Play,” Google. [Online].
Available:  \url{ https://play.google.com/store/apps/details?id=com.disasterprediction.ios&hl=en_CA.}
[Accessed: 07-Feb-2020]. \\

[2]“MyFireWatch - Bushfire map information Australia”, Myfirewatch.landgate.wa.gov.au,
 2020. [Online]. Available: \url{ https://myfirewatch.landgate.wa.gov.au/map.html. }
 [Accessed: 07- Feb- 2020].

\subsection{Overview}

The remainder of the document will consist of previous works and 
their purposes, the primary function of the software, user characteristics 
and the software’s target audience, constraints on the software design, 
and finally assumptions and dependencies in the project. This document is 
formatted in standard SRS 1998 format.



\section{Overall Description}
\subsection{Product perspective}
    \subsubsection*{Disaster Prediction App[1]} :
    This mobile application reports space weather changes as well as earthquake 
    incidents around the world. It shares similar characteristics with this 
    project such as reporting recent weather trends (albeit on the interplanetary 
    level) and this project as it is only concerned with solar influences and 
    earthquakes, but not storms, blizzards, or tornados, etc., which are more 
    relevant to daily lives. A way to improve it will be to include data reports on 
    hazardous weather that are more often concerned by people, like those 
    mentioned beforehand.
    \subsubsection*{MyFireWatch[2]} :
    MyFireWatch is a web service that visualizes recent fires by projecting 
    their location on a map. Similar to the proposed project, it highlights 
    regions most recently and often affected by natural hazards. However, it is 
    limited to just visualizing incidences of fire, not analyzing if there is any
    relation between the incidences’ geographic location and occurrence time, 
    which may be implemented to enhance the service’s usefulness.


\subsection{Product function}
The main function that the software will perform is visualizing the danger 
of a certain geographical region, providing guidance to avoid such 
hazards. The geographical location will be provided by the user 
through the use of a mobile GPS device or the manual input of positional 
coordinates.   

\subsection{User characteristics}
The intended users of this product will need minimal knowledge of the 
operation of a mobile device. The target audience will be mainly directed 
to government and rescue personnel.

\subsection{Constraints}
The reliability of the data is constrained, due to the non-infinity set 
of natural disasters recorded in the dataset. Another constraint with 
this data set is the limited set of data points relating to time.  

\section{Specific Features}
\subsection{Information Display}

\subsection{Design Constraints}
The product design must be consistent with Mcmaster Software Engineering 2XB3 Software Engineering Practice and Experience:Binding Theory to Practice Final project documentation.

\section{Software System Attributes}
\subsection{Maintainability}
Using Git issue tracker to allow the ease of maintablity and process tracker. The product will use separation of concerns along with modularity to allow for ease of maintenance.
\subsection{Portability}
Through the use of java and the product will be allowed to run on any system that supports Java 8 and above.


\end{document}