% !TEX root = ./Requirment.tex
\documentclass{article}
\usepackage[utf8]{inputenc}
\usepackage{url} 
\title{Sections and Chapters}
\author{Christina Korsman,Oussama Saoudi}
\date{\today}
  
\begin{document}
\begin{titlepage}
    \begin{center}
        \vspace*{1cm}
            
        \Huge
        \textbf{Software Requirement Specification}

        \LARGE
        \vspace{0.5cm}
        \textbf{Project Vayu}\\
        \vspace{0.2cm}
        Lab 2, Group 5

        \vspace{0.2cm}
        \today
            
        \vspace{1.5cm}
            
        \Large
        Oussama Saoudi, Lennon Yu\\
        Christina Korsman, Diego Soriano

        \vfill
            
        \vspace{0.8cm}
                        
        \large
        SFWRENG 2XB3\\
        Software Engineering Practice and Experience:\\
        Binding Theory to Practice\\
        Department of Computing and Software\\
        McMaster University            
    \end{center}
\end{titlepage}
  
\tableofcontents
\pagebreak
 
\section{Introduction}
\subsection{Purpose}
    This document is intended to act as a software requirements specification 
    for the program, which will outline the requirements that the software 
    must meet and the priority of each requirement. \\\\
The intended audience for the document is the client requesting the software, 
who will use it to understand the formalized specification for the software 
project being developed. The software developers will also use the document 
in the development of components to ensure that they meet the requirements specified.
 
\subsection{Scope}
	The software being developed is called Project: Vayu.\\\\
 
 Project: Vayu aims to provide information on natural disasters, 
 particularly information on property damage in sorted order, and 
 geographical locations affected by particular natural disasters. 
 The software will not add new data.\\\\
 
The objective of the project is to present natural disaster information in forms relevant to
disaster relief organizations and governments. This will be done by presenting regions prone to
certain disasters, what disasters they are prone to, which regions are most affected by natural disasters, 
and their property costs and estimated casualty count. The representation can be 
modified depending on the time of year, allowing for 
resource management of relief to be allocated to the right places 
at the right times of the year.
 
\subsection{Definitions, acronyms, and abbreviations}
SRS - Software Requirement Specifications

\subsection{References}
[1]“Disaster Prediction App – Apps on Google Play,” Google. [Online]. 
Available: \url{ https://play.google.com/store/apps/details?id=com.disasterprediction.ios&hl=en_CA.} [Accessed: 07-Feb-2020]. \\\\
 
[2]“MyFireWatch - Bushfire map information Australia”, Myfirewatch.landgate.wa.gov.au, 2020. [Online].
 Available: \url{ https://myfirewatch.landgate.wa.gov.au/map.html.} [Accessed: 07- Feb- 2020].

 \subsection{Overview}
 The remainder of the document will consist of previous works and their purposes, 
 the primary function of the software, user characteristics and the software’s 
 target audience, constraints on the software design, and finally assumptions and 
 dependencies in the project. This document is formatted in standard SRS 1998 format.

  
\section{Overall Description}
\subsection{Product perspective}
\subsubsection*{Disaster Prediction App[1]} 
    This mobile application reports space weather changes as well as earthquake 
    incidents around the world. It shares similar characteristics with this project 
    such as reporting recent weather trends (albeit on the interplanetary level) and 
    this project as it is only concerned with solar influences and earthquakes, but 
    not storms, blizzards, or tornados, etc., which are more relevant to daily lives. 
    A way to improve it will be to include data reports on hazardous weather that are 
    more often concerned by people, like those mentioned beforehand.
 
\subsubsection*{MyFireWatch[2]} 
    MyFireWatch is a web service that visualizes recent fires by projecting their 
    location on a map. Similar to the proposed project, it highlights regions most 
    recently and often affected by natural hazards. However, it is limited to just 
    visualizing incidences of fire, not analyzing if there is any relation between 
    the incidences’ geographic location and occurrence time, which may be implemented 
    to enhance the service’s usefulness.

 
    \subsection{System Interfaces}
	\subsubsection{User Interfaces}
    User interface will consist of a toolbar at the top and a window beneath it. 
    The toolbar will contain buttons to navigate through the software and perform 
    any functions available to the user. The window beneath will either have a map to 
    display the disaster regions, or will have a list to display the areas most 
    affected by the disasters through property damage.
 
	\subsubsection{Hardware Interfaces}
		No external hardware interfaces are required for this project.
 
	\subsubsection{Software Interfaces}
        Other than the standard Java project, this software does not 
        interface with any other software products.
 
	\subsubsection{Communications Interfaces}
		This project does not interface with any communications systems.
 
    \subsection{Product function}
        The main function that the software will perform is visualizing the 
        danger of a certain geographical region, providing guidance to avoid such 
        hazards. The geographical location will be provided by the user through the use 
        of a mobile GPS device or the manual input of positional coordinates.   
         
    \subsection{User characteristics}
         
        The intended users are professionals in disaster relief, or other professionals 
        who require detailed information on historic weather and disasters to either 
        prepare for future disasters or deal with the aftermath.
     
\subsection{Constraints}
The reliability of the data is constrained, due to the non-infinity set of natural disasters recorded 
in the dataset. Another constraint with this data set is the limited set of data points relating to time.
The format accepted for input data files shall only be of Comma Separated Value type.
 
 
\section{Specific Features}

\subsection{Functional Requirements}
    \begin{itemize}
        \item The system shall provide the user with a list of affected areas sorted by property damage cost.
        \item The system shall provide the user with a list of affected areas sorted by casualties
        \item The system shall provide the user with the list of affected locations with particular disaster types filtered out.
        \item The system shall output a set of convex hulls representing regions affected by a particular type of disaster.
        \item The output file shall be a text file
        \item The output formal shall be CSV (Comma Separated Value)
        \item The system shall Display the convex hull of each affected area superimposed on a map.
        \item The map display shall allow the filtering of disaster types.
        \item The map display shall allow the filtering of disasters based on time.
        \item The sorting of locations shall take no more than 15 minutes 95\% of the time.
        \item User-submitted reviews on interface compatibility and ease of use must be at least 75\% positive.
        \item The convex hulls representing disaster affected areas will cover at minimum 100\% of the areas affected by the disaster type.The area unaffected by the disaster that may be covered by the convex will be at most 50\% of the area of the convex hull.
    
    \end{itemize}
    \subsection{Design Constraints}
    The product design must be consistent with Mcmaster Software Engineering 
    2XB3 Software Engineering Practice and Experience: Binding Theory to Practice 
    Final project documentation.
    
    \section{Development and Maintenance Process Requirements}
    \subsection{System Test Procedures}
    \subsubsection{Stress Test}
    A stress test consisting of 200 thousands rows of data entries will be 
    performed upon the finished software, and the software shall not crash 
    or freeze during runtime unless external factors are involved in the 
    stress test (e.g. power outage, physically interfering the hardware),
    or the performance of the hardware is below minimum requirement.

    \subsubsection{Correctness Test}
    The convex hulls representing disaster affected areas will be validated 
    by checking that every disaster occurrence point of that type falls within a 
    corresponding convex hull. If all points fall within a convex hull of their 
    disaster type, then the software will be deemed correct.

    \subsubsection {Performance Test}
    A performance test consisting of recording the amount of time it takes to 
    process the data for finding and creating the convex hulls, and amount of 
    time it takes to sort the data.
        \subsection{Functional Requirements Priorities}
            \begin{itemize}
                \item Sort Locations Based on Damage: Essential
                \item Output Affected Regions to File: Essential
                \item Display Affected Regions on Map: Conditional
            \end{itemize}
        \subsection{Likely Changes to System Maintenance Procedure}
            Changes to system maintenance procedure may include but are not limited to the following:
    \begin{itemize}
    \item migration of the project to different version control providers.
    \item adding the use of issue trackers for bug fixing and functionality improvements.
    \item disallowing direct modification of the source code and only making changes by creating new files.
    \end{itemize}
    



\section{Software System Attributes}
\subsection{Maintainability}
Using Git issue tracker to allow the ease of maintablity and process tracker. The product will use separation of concerns along with modularity to allow for ease of maintenance.
\subsection{Portability}
Through the use of java and the product will be allowed to run on any system that supports Java 8 and above.

\end{document}